\documentclass{article}
\usepackage[utf8]{inputenc}
\usepackage{amsmath}
\usepackage{hyperref}

\title{PHYS 272: Pre-Lecture Notes and Functions}
\author{Nicholas Grogg}

\begin{document}

\maketitle

\section{Electricity}
BEGIN TEST 1
\subsection{Lecture 1: Coulomb's law}

\subsubsection{Pre-Lecture}
\noindent
Coulomb's Law: \\
$k = 9*10^9 \frac{Nm^2}{C^2}$ \\
$\vec{F}_{12} = k\frac{q_1 q_2}{r^{2}_{12}}\hat{r}_12$\\
REMEMBER, convert from $\mu$C to C! 1C = 1,000,000 $\mu$

$\vec{F} = G\frac{m_1 m_2}{r^2}\hat{r}$

\vspace{2mm}

\noindent
Superposition principle\\
$\vec{F} = q\vec{E} + q\vec{v} * \vec{B}$

\vspace{2mm}

\subsection{Lecture 2: Electric Fields}

\subsubsection{Pre-Lecture}
\noindent 
$\vec{F} = k*\frac{Qq}{r^2}\hat{r}$

\vspace{2mm}
\noindent
Discrete Distribution \\
$\vec{E} \equiv \frac{\vec{F}}{q} = k \frac{Q}{r^2}\hat{r}$ \\
$\vec{E} \equiv \frac{\vec{F}}{q} = \frac{1}{q} \sum^{}_{i} \vec{F}_{iq}$ 
$\vec{E} = k \sum^{}_{i} \frac{Q_i}{r^2_{iq}} \hat{r}_iq$

\vspace{2mm}
\noindent
Continuous Distribution \\
$\vec{E} \equiv \frac{\vec{F}}{q} = \frac{1}{q} \int d\vec{F} = k \int \frac{dQ}{r^2}\hat{r}$

\subsubsection{Homework}

\noindent
I can't find my work for Question 1, sorry...

\vspace{2mm}

\noindent
Question 2-1: $E_x(P)$ \\
$E_x(P) = k \frac{q_1}{r^2} cos(45)$

\vspace{2mm}

\noindent
Question 2-2: $E_y(P)$\\
$E_y(P) = E_x(P) + k \frac{q_2}{d^2} $

\vspace{2mm}
Question 2-3: A third charge is added \\
$E_x(P) = k(\frac{q_1}{2d^2} cos(45) + \frac{q3}{d^2}) $

\vspace{2mm}

\noindent
Question 2-4 \\
I guessed C, I don't have the work for it sorry...

\vspace{2mm}

\noindent
Question 2-5 \\
I guessed E, I don't have the work for it sorry...

\subsection{Lecture 3: Electric Flux and Field Lines}

\subsubsection{Pre-Lecture}
\noindent
Electric Flux \\
$\phi \equiv \int \vec{E} * d\vec{A}$ \\
$\varepsilon_0 = 8.85 * 10^-12$

\subsubsection{Homework}

\noindent
Question 1-1: What is $E_x(P)$ \\
E = 2k$\frac{\lambda_1}{a}$ \\
Conversion for $\lambda$ is $\lambda * 10^{-4}$ \\
For mine, that means $\lambda_1 = -2.2$ is used as $-2.2 * 10^-4$ \\
That converts from $\frac{\mu C}{cm}$ to $\frac{C}{m}$ \\
I looked it up...

\vspace{2mm}

\noindent
Question 1-2: $E_y(P)$ \\
There is no Y movement, so the answer is 0.

\vspace{2mm}

\noindent
Question 1-3: Total Flux pt.1 \\
$\phi = h \frac{\lambda_1}{\varepsilon_0}$

\vspace{2mm}

\noindent
Question 1-4: What is the new value for $E_x(P)$? \\
$E_x(P) = \frac{2}{4 \pi \varepsilon_0} (\frac{\lambda_1}{a} + \frac{\lambda_2}{x})$ \\
x = $\frac{a}{2}$

\vspace{2mm}

\noindent
Question 1-5: total flux $\phi$ pt.1 \\
I fucked this one up too many times with typos, but I'm pretty sure it's \\
$h (\frac{\lambda_1 + \lambda_2}{\varepsilon_0}$) \\
This is also the answer for 7

\vspace{2mm}

\noindent
Question 1-6: Total Flux pt.2 \\
$2k (\frac{\lambda_2}{a} + \frac{\lambda_1}{\frac{3a}{2}}) $

\vspace{2mm}

\noindent
Question 1-7: Total flux $\phi$ pt.2 \\
Same function as 1-5, hopefully...

\vspace{2mm}

\subsection{Lecture 4: Gauss' Law}

\subsubsection{Pre-Lecture}
\noindent
Gauss' Law \\
$\phi_{Net}$ = $\oint \vec{E}*d\vec{A}$ = $\frac{q_{enclosed}}{\varepsilon_0}$ \\
$\phi = \frac{q_1}{\varepsilon_0} + \frac{q_2}{\varepsilon_0} $

\vspace{2mm}

\noindent
On conducting shell\\
$Q_{inner} = -q_o$ \\
Induced inner charge density \\
$\sigma_i = \frac{-q_o}{4 \pi R^{2}_{i}}$ \\
Out charge density \\
$\sigma_o = \frac{Q + q_o}{4 \pi R^{2}_{o}}$

\vspace{2mm}

\noindent
Gauss' Law on a Sphere \\
E = $\frac{Q}{4 \pi \varepsilon_0 r^2}$ \\
Gauss' Law on a Cylinder \\
E = $\frac{\lambda}{2 \pi \varepsilon_0 r}$ \\
Infinite sheet of charge \\
E = $\frac{\sigma}{2 \varepsilon_0}$

\vspace{2mm}

\noindent
Spherical (3d) Field line density \\
$\frac{~1}{A_{sphere}} (~\frac{1}{r^2})$ \\
Cylindrical (2d) Field line density \\
$\frac{~1}{A_{cylinder}} (\frac{~1}{r})$ \\
Planar (1d) Field line density \\
Constant

\subsubsection{Homework}

\noindent 
Question 1 \\
E = $\frac{q_{enclosed}}{A_{sphere} \varepsilon_0}$ \\
$q_{enclosed} = \frac{Q}{4 \pi 100 (b^2 - a ^2)}$ \\
$A_{sphere} = 4 \pi r^2$ \\
Since the answer seems to be the same no matter what \\
Answer =  -1.28798 * $10^7 \frac{N}{C}$

\vspace{2mm}

\noindent
Question 2 \\
$E = \frac{x}{2 \pi \varepsilon_0 r}$ \\
$E_x = \frac{\frac{Q_inner}{L}}{2 \pi \varepsilon_0 r}$ \\
Again, answer seems to be the same \\
Answer = $311.485 \frac{N}{C}$

\vspace{2mm}

\noindent
Question 3 \\
3-1 \\
$E_x(P) = \frac{q_1 + q_2}{r^2}$
3-2 \\
$E_y$(P) = 0 \\
3-3 \\
$E_x(R)$ = 0 \\
3-4 \\
$E_y(P) = k\frac{q_1}{r^2}$ \\
3-5 \\
$\sigma_b = \frac{q_1 + q_2}{4 \pi b^2}$ \\
3-6 \\
$\sigma_a = \frac{q_1}{4 \pi a^2}$ \\
3-7 \\
A, none\\
Field is treated as if it's a single point. \\
Really though, we have 3 choices and 5 guesses \\
Guess until it's right! \\
3-8 \\
B, $E_2 = E_0$ \\
Fields are equal as the charge on the outer shell has no effect on field in shell. \\
Again though, 3 choices 5 chances, throw a dart!

\vspace{2mm}

\noindent
Question 4 \\
4-1 \\
$\lambda_2 = \rho * \pi (b^2 - a^2)$ \\
4-2 \\
Answer = 0 \\
4-3 \\
$E_y(P) = \frac{(\lambda_1*10^6 + \lambda_2*10^6)}{2 \pi \varepsilon_0 r}$ \\
4-4 \\
E = ($\frac{\lambda_1*10^6}{2 \pi \varepsilon_0 r}$) \\
4-5 \\
E = ($\frac{\lambda_1*10^6}{2 \pi \varepsilon_0 r}$) \\
4-6 through 4-8 \\
a, b, and d

\vspace{2mm}

\noindent
Question 5 \\
5-1 \\
E = $\frac{\lambda_1*10^6 + \lambda_2*10^6}{2 \pi \varepsilon_0 P}$ \\
5-2 \\
Answer is 0 \\
5-3 \\
E = $(\frac{\lambda_1*10^6}{2 \pi \varepsilon_0 r}) sin(30)$ \\
5-4 \\
E =  $(\frac{\lambda_1*10^6}{2 \pi \varepsilon_0 r}) cos(30)$ \\
5-5 \\
$\lambda_1 + \lambda_2$ \\
5-6 \\
- $\lambda_1$ \\
5-7 \\
$\lambda_1$

\vspace{2mm}

\noindent
Question 6 \\
6-1 \\
E = $\frac{\sigma_1 + \sigma_2}{2 \varepsilon_0}$ \\
6-2 \\
Answer = 0 \\
6-3 \\
E = $\frac{\sigma_1-\sigma_2}{2 \varepsilon_0}$ \\
6-4 \\
Answer = 0 \\
6-5 \\
$\sigma_b = \frac{\sigma_1 + \sigma_2}{2}$ \\
6-6 \\
Answer = 0 \\
6-7 \\
$\sigma_a = \frac{\sigma_2 - \sigma_1}{2}$ \\
6-8 \\
None 

\subsection{Lecture 5: Electric Potential Energy}

\subsubsection{Pre-Lecture}
\noindent
Coulomb Force, conservative force \\
$\vec{F}_E = \frac{1}{4 \pi \varepsilon_0} \frac{Qq}{r^2} \hat{r}$

\vspace{2mm}

\noindent 
Work done by Coulomb Force \\
$\vec{F}_E = \frac{1}{4 \pi \varepsilon_0} \frac{q_1 q_2}{r^2} \hat{r}$ \\
$W_{A \rightarrow B} = \int_{r_A}^{r_B} \vec{F}_E * d\vec{r}$ \\
$W_{A \rightarrow B} = \frac{q_1 q_2}{4 \pi \varepsilon_0} \int_{r_A}^{r_B} \frac{1}{r^2} dr$ \\
$W_{A \rightarrow B} = \frac{q_1 q_2}{4 \pi \varepsilon_0} (\frac{1}{r_A} - \frac{1}{r_B} )$ 

\vspace {2mm}

\noindent
Electric Potential Energy \\
$\Delta U_{AB} = W_{A \to B}$ \\
Often use $r_A = \infty$ \\
End up with \\
$U_r \equiv \Delta U_{\infty r} = \frac{q_1 q_2}{4 \pi \varepsilon_0 r} $

\vspace{2mm}

\noindent
Calculate Speed \\
v = $\sqrt{\frac{q_1 q_2}{2 \pi \varepsilon_0 m_2}(\frac{1}{d} - \frac{1}{x})}$ \\
$v_{max} = \sqrt{\frac{q_1q_2}{2 \pi \varepsilon_0 m_2 d}} $

\vspace{2mm}

\noindent
System of Three Particles \\
$\Delta U_1 = 0$ \\
$\Delta U_2 = k \frac{q_1 q_2}{d}$ \\
$\Delta U_3 = k \frac{q_1 q_3}{d} + k \frac{q_2 q_3}{d}$ \\
$U_{System} = \Delta U_1 + \Delta U_2 + \Delta U_3$ \\
System of N charged Particles \\
$U_{System} = k \sum^{}_{i<j} \frac{q_i q_j}{r_{ij}} $

\vspace{2mm}

\noindent 
Lecture slides \\
$\Delta U = +\frac{1}{4 \pi \varepsilon_0} \frac{Qq}{r} - \frac{1}{4 \pi \varepsilon_0} \frac{2Qq}{r+d}$ \\
For charges Q $\to$ d $\to$ Q $\to$ r Q \\
$U = U_i + k qxq(\frac{1}{r} - \frac{2}{d+r}) = 0$ \\
For charges $Q_1 \to r Q Q_2$, with distance d between charges $Q_1 and Q_2$ \\
$U = U_i + k\frac{qxq}{r} - k \frac{qx2x}{d-r}$ \\
I don't remember what the x's represent in this case...

\subsubsection{Homework}

\noindent
5-1 \\
$\Delta PE = \frac{q_1 q_2}{4 \pi \varepsilon_0}(\frac{1}{d_2}- \frac{1}{d_1})$

\vspace{2mm}

\noindent
5-2 \\
$\Delta U = 2(k \frac{q_1 q_3}{r_2} - k \frac{q_1 q_2}{r_1})$ \\
$r_1 = \sqrt{a^2 + d^2_1}$ \\
$r_2 = \sqrt{a^2 + d^2_2}$

\vspace{2mm}

\noindent
5-3 \\
U = $k \frac{q_a q_b}{r^2} + k \frac{q_c q_d}{2a}$

\vspace{2mm}

\noindent
5-4 \\
U = k $\frac{q_3 q_5}{2a}$

\vspace{2mm}

\noindent
5-5 \\
Answer is 0
\subsection{Pre-Lecture 6: Electric Potential}

\subsubsection{Pre-Lecture}

\noindent
Electric Potential \\
V $\equiv \frac{U}{q} $ \\
$\vec{E} \equiv \frac{\vec{F}}{q} = k \frac{Q}{r^2}\hat{r}$

\vspace{2mm}

\noindent
Electric Potential Energy \\
$W_{A \rightarrow B} = \int_A^B \vec{F} * d\vec{l} $ \\
$\Delta U_{A \to B} = -W_{A \to B} $

\vspace{2mm}

\noindent
Electric Potential Difference \\
$\Delta V_{A \to B} \equiv \frac{\Delta U_{A \to B}}{q} = \int_{A}^{B} \vec{E} * d\vec{l}$ \\
$\Delta V_{A \to B} = kQ (\frac{1}{r_B} - \frac{1}{r_A})$

\vspace{2mm}

\noindent
Electric Potential \\
V(r) $\equiv \Delta V_{r_0 \to r} = kQ(\frac{1}{r} - \frac{1}{r_0})$

\vspace{2mm}

\noindent
Electric Potential for Point Charge \\
V(r) = $\frac{kQ}{r}$

\vspace{2mm}

\noindent
The Gradient in different coordinate systems \\
Cartesian \\
$\vec{\nabla}V = \frac{\partial V}{\partial x}\hat{i} + \frac{\partial V}{\partial y}\hat{j}+\frac{\partial V}{\partial z}\hat{k}$ \\
Spherical \\
$\vec{\nabla}V = \frac{\partial V}{\partial r}\hat{r} + \frac{1}{r}\frac{\partial V}{\partial \theta}\hat{\theta} + \frac{1}{r sin(\theta)}\frac{\partial V}{\partial \phi}\hat{\phi}$ \\
Cylindrical \\
$\vec{\nabla}V = \frac{\partial V}{\partial r}\hat{r} + \frac{1}{r}\frac{\partial V}{\partial \theta}\hat{\theta} + \frac{\partial V}{\partial z}\hat{k}$

\vspace{2mm}

\noindent
$V_{Total} = \sum^{}_{i} V_i$ \\
$V_p = k\frac{q}{a}(2-\frac{\sqrt(2)}{2})$

\vspace{2mm}

\noindent
if $(r < a)$ \\
E = k $\frac{Q}{a^3}r$ \\
if $(r > a)$ \\
E = k $\frac{Q}{r^2}$

\vspace{2mm}

\noindent
Find V(r) \\
For $r > a$ \\
$V(r) = k \frac{Q}{r}$ \\
For $r < a$ \\
$V(r) = k \frac{Q}{2a^3}(3a^2 - r^2) $

\subsubsection{Homework}

\noindent
1-1 \\
Spheres V \\
$\int_{\infty}^{0} E dr = -V(0)$ \\
$\int_{\infty}^{0} E dr = \int_{\infty}^{9cm} + \int_{2.5cm}^{6cm} E dr $\\
$\int E dr = \frac{-1}{4 \pi r \varepsilon_0}$ \\
99.86 V + 629.13 V = 728.99V \\
Factor in direction, answer $\approx$ -729V


\vspace{2mm}

\noindent
2-1 \\
I made a power of 10 error here \\
Answer is something along  \\
$Q = \rho * \pi * \frac{4}{3} * a^3$ \\
$E_x = \frac{kQ}{r^2}$ \\
I fucked up the unit conversion somewhere...

\vspace{2mm}

\noindent
2-2 \\
V = $\frac{KQ}{c}$ \\
Remember to convert $K*10^{-6}$ 

\vspace{2mm}


\noindent
2-3 \\
$V_b + kQ(\frac{1}{r_a} - \frac{1}{r_b})$

\vspace{2mm}


\noindent
2-4 \\
Absolute value of V(a) -V(b) 

\vspace{2mm}


\noindent
2-5 \\
I did it wrong, but essentially you redo part 2\\
$V_b = \frac{kQ_{new}}{c}$ \\
Then add it to the value for 4 \\
I kept getting minor errors.

\vspace{2mm}

\noindent
3-1 \\
$\lambda_{inner} = \rho \pi a^2$ \\
$\lambda_{enclosed} = \lambda_{inner} + \lambda_{outer}$ \\
E = 2k$\frac{\lambda_{enclosed}}{d}$ 

\vspace{2mm}

\noindent
3-2 \\
-($\frac{\lambda_{enclosed}}{2 \pi \varepsilon_0} ln(P) - \frac{\lambda_{enclosed}}{2 \pi \varepsilon_0} ln(R)$) 

\vspace{2mm}


\noindent
3-3 \\
$\frac{\rho a^2}{2 \varepsilon_0} ln(\frac{b}{a})$ \\
I have no idea why that's correct, but it is...

\vspace{2mm}


\noindent
3-4 \\
A, $V(a) < 0$ \\
\vspace{2mm}


\noindent
3-5 \\
$\rho = -\frac{\lambda_{outer}}{\pi a^2}$

\vspace{2mm}

\noindent
4-1 \\
E = $\frac{\sigma_1}{2 \varepsilon_0} + \frac{|\sigma_2|}{2 \varepsilon_0}$
\vspace{2mm}

\noindent
4-2 \\
$-\frac{|\sigma_i + \sigma_2|}{2} = \sigma_a$

\vspace{2mm}

\noindent
4-3 \\
0

\vspace{2mm}

\noindent
4-4 \\
Bear with me on this one\\
E*$((S_x-R_x) - (b_x-a_x))$ \\
Where $S_x$ etc. is the x value for each point

\vspace{2mm}

\noindent 
4-5 \\
E = $\frac{\sigma_1}{2 \varepsilon_0} - \frac{|\sigma_2|}{2 \varepsilon_0}$

\vspace{2mm}

\noindent
4-6 \\
B

\subsection{Pre-Lecture 7: Conductors and Capacitance}

\subsubsection{Pre-Lecture}

\noindent 
Capacitance \\
C $\equiv \frac{Q}{\Delta V}$

\vspace{2mm}

\noindent 
$\Delta V = \frac{Q}{\varepsilon_0 A} d$ \\
For parallel-plates \\
C = $\frac{\varepsilon_0 A}{d}$

\vspace{2mm}

\noindent
dU = Vdq \\
Stored Energy difference \\
U = $\frac{1}{2}QV = \frac{1}]{2} \frac{Q^2}{C} = \frac{1}{2} CV^2 $ \\
U = $\frac{1}{2} \varepsilon_0 E^2 Ad$ \\
Energy density in area between plates \\
u $\equiv \frac{U}{aD} = \frac{1}{2}\varepsilon_0 E^2$
General energy density \\
u = $\frac{1}{2}\varepsilon_0 E^2$

\vspace{2mm}


\subsubsection{Homework}
\noindent
Please note when converting $cm^2$ to $m^2$ the conversion rate is $cm^2\times 10^{-4}=m^2$ \\
1-1 \\
$C= \frac{A\varepsilon_0}{d}$ \\
Q = C$V_b$ \\
My answer was on the order of $10^{-9} C$
\vspace{2mm}

\noindent
1-2 \\
$U=\frac{QV}{2}$ \\
My answer was on the order of  $10^{-8}J$
\vspace{2mm}

\noindent
1-3 \\
$U=QV$ \\
Moving the plates apart does work and changes (increases) the voltage. Double distance = double voltage.

\vspace{2mm}

\noindent
1-4 \\
$\sigma = \frac{Q}{A}$ \\
$E= \frac{\sigma}{\varepsilon_0}$ \\
My answer was on the order of $10^{3}\frac{N}{C}$
\vspace{2mm}

\noindent
1-5 \\
$V>V_b$

\vspace{2mm}

\noindent
1-6 \\
Both $E$ and $V$ decrease

\vspace{2mm}

\section{DC Circuits}

\subsection{Section 8: Capacitors}
\subsubsection{Pre-Lecture}
\noindent
Capacitance of two parallel-plates = $\frac{a*b}{d}$ \\
Dielectric increases Capacitance and reduces electric field \\
Dielectric constant $\kappa$ \\
$C_{Dielectric} = \kappa C_0$ \\
U = $\frac{1}{2}QV_C$

\vspace{2mm}

\noindent
Parallel processing is basically a sum of the capacitors \\
$C_{equivalent} = \frac{\varepsilon_0 A{equivalent}}{d} = \frac{\varepsilon_0 (A_1 + A_2)}{d} $ \\
C= $C_1 + C_2$ \\
Series \\
$\frac{1}{C_1} + \frac{1}{C_2} = \frac{1}{C_{equivalent}}$ \\
$\frac{1}{C_{equivalent}} = \frac{d_1 + d_2}{\varepsilon_0A}$


\subsubsection{Homework}

\noindent
1-1 \\
1 NC = $10^9C$ \\
One cap is 10 nF, the other is 10x2.6nF. Total of 36nF \\
36nF * 12 V = 432 nC = $4.32*10^-7$

\vspace{2mm}

\noindent
2-1 \\
$C_3 + C_4 = \frac{2}{3}$ \\
In parallel with $C_5 and C_6 = \frac{20}{3}$ \\
In series witrh $C_1 = \frac{3 * (\frac{20}{3})}{(3 + \frac{20}{3})} = \frac{60}{29} = 2.069 $ \\
Q = CV = 12 * 2.069 = 24.8 $\mu$C \\
$V_{AB} = \frac{24.8}{\frac{20}{3}}$ = 3.72 V \\
voltage across $C_3 and C_4$ together \\
charge on these two is Q = 3.72 * $\frac{2}{3}$ = 2.48 $\mu$ C \\
Voltage across $C_4$ = $\frac{Q}{C} = \frac{2.48}{1}$ = 2.48 V

\vspace{2mm}

\noindent
3-1 \\
$C_{23} = (\frac{1}{C_2} + \frac{1}{C_3})^{-1}$ \\
$C_{ab} = C_{23} + C_4$ \\
3-2 \\
$C_{ab} = (\frac{1}{C_1} + \frac{1}{C_{ab}})^{-1}$ \\
3-3 \\
$Q_5 = C_{total}V$ \\
3-4 \\
$V_{ab} = V - 2\frac{Q_5}{C_5}$ \\
$Q_2 = V_{ab} * \frac{C_2 * C_3}{C_2 + C_3}$ \\
3-5 \\
$Q_5 = C_{total}V$ \\
3-6 \\
$V_4 = V_{ab} = V - 2\frac{Q_5}{C_5}$

\vspace{2mm}

\noindent
4-1 \\
MIND YOUR UNITS HERE! I shit the bed SO bad on this section! \\
C = $\varepsilon_0 \frac{A}{d}$ \\
Convert $cm^2$ to $mm^2$ and cm to mm \\
$cm^2$ to $mm^2$, add two zeroes! \\
4-2 \\
Q = C*Vb \\
4-3 \\
$Q_{new} = Q * \frac{2\kappa}{1+ \kappa}$ \\
4-4 \\
Convert $\mu C to C$ \\
1 $\mu C = 10^{-6} C $ \\
$U_{new} = \frac{Q_{new} V_b}{2}$ \\
4-5 \\
V = $\frac{Q_new}{C}$

\vspace{2mm}

\noindent 
5-1\\
$E_x(P) = 2k \frac{\lambda_{inner}}{d}$ \\
5-2\\
Answer should be positive! Mind the negative!!!\\
- $\frac{-\lambda_{inner}*10^-6}{2 \pi \varepsilon_0} ln(\frac{b}{a})$ \\
5-3\\
I fucked up the unit conversion, should be answer$^{-5}$ \\
In my case it's $6.55679*10^{-5}$\\
$\frac{1}{2*k*ln(\frac{b}{a})}$ \\
5-4\\
B, just guess \\
5-5\\
$\lambda_{outer,new} = \lambda_{outer} * 2$

\subsection{Section 9: Electric Current}
\subsubsection{Pre-Lecture}

\noindent 
Electric Current  \\
I $\equiv \frac{dq}{dt} $ \\
Ohm's Law \\
J = $\sigma E$ \\
Ampere A = $\frac{Coulomb (C)}{second (S)} $\\
Current Density J $\equiv \frac{I}{A} = n_e e v_{drift}$ \\
$n_e = N_A \frac{\rho_{mass}}{M}$ \\
J $\propto$ E \\
$\sigma$ evidently means conductivity now \\
$v_{drift} = \frac{\sigma}{n_ee}E$ \\
Resistance \\
R $\equiv \frac{1}{\sigma} \frac{L}{A}$ \\
J $\equiv \sigma \frac{V}{L} $ \\
R = $\rho \frac{L}{A}$ \\
V = IR

\vspace{2mm}

\noindent
$V_{AB} = V_1+V_2 $ \\
$R_{quivalent} = R_1 + R_2$ \\
Power = IV = $I^2$R

\subsubsection{Homework}

\noindent 
1-1 \\
$I_1 = \frac{V}{R_1 + R_3}$ \\
1-2 \\
$V_2 = V_1 = I_1 R_1 = V * \frac{R_1}{R_1 + R_3}$ \\
1-3 \\
$I_2 = \frac{V_2}{R_2}$ \\
1-4 \\
$R_x = R_2 \frac{R_3}{R_1}$ \\
1-5 \\
$V_1 = V_2$ \\
1-6 \\
B

\vspace{2mm}

\noindent
2-1 \\
$ R_{23} = R_{ab} = \frac{R_4 (R_2 + R_3)}{R_4 + R_2 + R_3}$ \\
2-2 \\
$R_{ac} = R_1 + R_{ab}$ \\
2-3 \\
$R_{equiv} = R_5 + R_{ac}$ \\
$I_5 = \frac{V}{R_{equiv}}$ \\
2-4 \\
$V_{ab} = V (\frac{R_{ac} - R_5}{R_{ac} + R_5})$ \\
$I_2 = \frac{V_{ab}}{R_2 + R_3}$ \\
2-5 \\
$I_1 = I_5$ \\
2-6 \\
$V_4 = V_{ab}$

\subsection{Section 10: Kirchoff's Rules}
\subsubsection{Pre-Lecture}
\noindent
Voltage Rule \\
$\sum \Delta V_n = 0$

\noindent
Current Rule \\
$\sum I_{in} = \sum I_{out}$ \\
$V_c = \frac{Q}{C}$ \\
$V_b = V_0 \frac{\frac{R}{r}}{1+ \frac{R}{r}}$

\subsubsection{Homework}

\noindent
1-1: Current Div \\
You only have to write $i_2$ in terms of $i_1$ \\
In this case is it's $i_2 = \frac{1}{3}i_1$ \\
Flipit requires it in the form (1/3)*i1

\vspace{2mm}

\noindent
2-1: Multiloop \\
$I_1 + I_2 + I_3 = 0$ \\
$I_2 = I_2 + I_3$ \\ 
$V_1 - I_1*R_1 - I_3*R_3 = 0$ \\
$v_2 + I_3*R_3 - I_2*R_2 = 0$ \\
If you're too lazy, just put .058...

\vspace{2mm}


\noindent
Section 3 \\
You first must solve this matrix (\href{https://matrixcalc.org/en/slu.html}{matrix solver}), it encodes the circuit using Kirchoff's Laws and allows us to solve for all of the currents simultaneously. \\
\href{https://i.imgur.com/vFWyAQU.png}{analyzed circuit} 
\[
    \left[\begin{array}{ccc|c}
    -R_1 & R_3+R_1 & 0 & -V_{s1} \\
    R_2+R_6 & R_3 & 0 & V_{s2}-V_{s1} \\
    0 & 0 & R_5+R_4 & V_{s2} \\
    \end{array}\right]=
    \begin{bmatrix}
      I_2\\I_3\\I_4\\
    \end{bmatrix}
\]
\noindent
3-1 \\
$V_4=I_4R_4$ \\
\vspace{2mm}

\noindent
3-2 \\
Solved via the matrix. \\
\vspace{2mm}

\noindent
3-3 \\
Solved via the matrix. \\
\vspace{2mm}

\noindent
3-4 \\
Kirchoff's node Law \\
$I_1=I_2-I_3$ \\
\vspace{2mm}

\noindent
3-5 \\
$V_{ab}=I_2R_6$ \\
\vspace{2mm}

\noindent
Same deal as problem 3. \\
\href{https://i.imgur.com/MnsQ44k.png}{analyzed circuit}
\[
    \left[\begin{array}{cc|c}
    R_1+R_2& -R_2&V\\
    -R_2&R_2+R_3+R_4+R_5&0\\
      \end{array}\right]=
      \begin{bmatrix}
        I_1\\I_3
      \end{bmatrix}
\]
\noindent
4-1 \\
Solved via the matrix. \\
$I_1\times 10^{3}$ (milliampere conversion) \\
\vspace{2mm}

\noindent
4-2 \\
r = $\frac{V - V_b}{I_1}$
\vspace{2mm}

\noindent
4-3 \\
Solved via the matrix. \\
$I_3\times 10^{3}$ (milliampere conversion) \\
\vspace{2mm}

\noindent
4-4 \\
$P_2=(I_1-I_3)^2R_2$ \\
\vspace{2mm}

\noindent
4-5 \\
$V_2=(I_1-I_3)R_2$ 

\subsection{Section 11: RC Circuits}
\noindent
BEGIN TEST 2!!
\subsubsection{Pre-Lecture}

\noindent
If $t = 0$ \\
$V_C$(0) = 0 \\
$I(0) = \frac{V}{R}$

\vspace{2mm}

\noindent
As t increases \\
q, $V_C$ increases \\
I, $V_R$ decreases

\vspace{2mm}

\noindent
As t $\to \infty$ \\
q $\to CV_b$ \\
I $\to 0$

\vspace{2mm}

\noindent
Question 1 \\
Charge flows into the top of the capacitor and out of the bottom of the capacitor \\
but no charge actually crosses the gap between the plates

\vspace{2mm}

\noindent
Kirchoff's Voltage Rule \\
IR + $\frac{q}{C} - V_b = 0$ \\
R$\frac{dq}{dt} + \frac{1}{C}q - V_b = 0$ \\
q(t) = C$V_b(1-e^{\frac{-t}{RC}})$ \\
I(t) = $\frac{V_b}{R} e^{\frac{-t}{RC}}$ \\
Boundary Conditions \\
q(0) = 0 \\
I(0) = $\frac{V_b}{R}$ \\
q$(\infty) = CV_b$ \\
I$(\infty) = 0$

\vspace{2mm}

\noindent
Discharging a Capacitor \\
t = 0 \\
$V_C(0) = \frac{q_0}{C}$ \\
q(t) = $q_0 e^{\frac{-t}{RC}}$
Boundary Conditions \\
q(0) = $q_0$ \\
q$(\infty) = 0$ \\
I(t) = -$\frac{q_0}{RC}e^{\frac{-t}{RC}}$
Boundary Conditions \\
$\vert I(0) \vert = \frac{q_0}{RC}$ \\
I$(\infty) = 0$

\vspace{2mm}

\noindent
Time Constant \\
$\tau = RC$ \\
$I(\tau) = I_0e^{-1} \approx I_0(0.37)$

\vspace{2mm}

\noindent
Question 2 \\
t=2, V = $\frac{V}{2}$ \\
t=6, V = $\frac{V}{8}$

\vspace{2mm}

\noindent
Power in an RC Circuit \\
Fuck this section for making me type this much... \\
$P_{Battery}(t) = V_b I_0 e^{\frac{-t}{RC}}$ \\
$P_R(t) = RI_0^2 e^{\frac{-2t}{RC}}$ \\
$P_C(t) = (\frac{q_0}{C}(1 - e^{\frac{-t}{RC}}))(I_0 e^{\frac{-t}{RC}})$

\subsubsection{Homework}

\noindent
Section 1  \\
Convert $\mu$C to C! \\
$Q_{2final} = \frac{\frac{R_2 V}{R_1 + R_2}}{\frac{1}{C_1} + \frac{1}{C_2}}$ \\
If your come down with a case of fuckit, Interactive example = .000213

\vspace{2mm}

\noindent
Section 2 \\
2-1 \\
$I_1(0) = \frac{V}{R_1 + R_4}$ \\
2-2 \\
$I_1(\infty) = \frac{V}{R_1 + R_2 + R_3 + R_4}$ \\
2-3 \\
$I = \frac{V}{R_1 + R_2 + R_3 + R_4}$ \\
$V = I*R_{23}$ \\
$Q = VC$ \\
2-4 \\
$I_1(0) = \frac{V}{R_1+R_{523}}$ \\
$R_{23} = R_2 + R_3 $ \\
$R_{235} = \frac{1}{\frac{1}{R_{23} + R_5}}$ \\
$I_1 = \frac{V}{R_1 + R_{523} + R_4}$ \\
2-5 \\
Same as 2-3

\vspace{2mm}

\noindent
Section 3 \\
3-1 \\
$I_4(0) = \frac{V}{R_{equiv}}$ \\
$R_{equiv} = R_1 + R_4 + \frac{R_2 R_3}{R_2 + R_3}$ \\
3-2 \\
$Q(\infty) = I(\infty) R_3 C$ \\
$Q(\infty) = CV \frac{R_3}{R_1 + R_3 + R_4}$ \\
3-3 \\
$\tau = (R_2 + R_3) C$ \\
$t_{open}$ is provided \\
$Q(t_{open}) = Q(\infty) e^{\frac{-t_{open}}{\tau}}$ \\
3-4 \\
$I_{c,max}(closed) = \frac{V - I_4(0) (R_1 + R_4)}{R_2}$ \\
3-5 \\
$I_{c,max}(open) = \frac{Q(\infty)}{(R_2 + R_3) C}$

\noindent

\section{Magnetism}
\subsection{Section 12: Magentism}
\subsubsection{Pre-lecture}
\noindent
Lorentz Force \\
$\vec{F}=q \vec{E} + q \vec{v} * \vec{B} $ \\
Fucking magnets man! How do they work? 

\noindent
Cross Product \\
$\vec{F} = q \vec{v} * \vec{B}$ \\
Force perpendicular to Current and magnetic field direction\\
$\vec{F} \bot \vec{I}$ \\
$\vec{F} \bot \vec{B}$

\noindent 
Cross Product \\
$\vert \vec{A} * \vec{B} \vert = AB sin \theta$

\noindent
$\vec{F}_{Electric} = -\vec{F}_{Magnetic}$ \\
$q\vec{E} = -q\vec{v} * \vec{B}$ \\
$F_B = qvB$ \\
$F_e = qE$ \\
$v = \frac{E}{B}$ when $\vec{F}_E = -\vec{F}_B$ \\
$a_c = \frac{v^2}{R}$ \\
R = $\frac{mv}{qB}$

\subsubsection{Homework}
\noindent
1-1 \\
$r = \frac{mv}{qB}$ \\ 
$t = \frac{\pi r}{v}$ \\
It's an Interactive example, answer = 2.73*$10^-8$ s

\vspace{2mm}

\noindent
2-1 \\
$\text{path} = \frac{1}{2}\pi d $ \\
$v =  \frac{\text{path}}{t} $ \\
Remember to convert $\mu$s to s \\
t*$10^-6$

\vspace{2mm}

\noindent
2-2 \\
$F =\frac{mv^2}{r} $ \\
$\theta  = \pi\left( \frac{t_{1}}{t}+1 \right) $ \\
$F_x =F \sin(\theta)$ 

\vspace{2mm}

\noindent
2-3 \\
$F =\frac{mv^2}{r}$\\
$\theta  = \pi\left( \frac{t_{1}}{t}+1 \right) $ \\
$F_y =F \cos(\theta)$ 

\vspace{2mm}

\noindent
2-4 \\
Sign must be ascertained from the cross product of the direction of the $\vec{B}$ and $\vec{v}$. \\
$q = \frac{F}{vB}$ 

\vspace{2mm}

\noindent 
2-5 \\
A

\vspace{2mm}

\noindent
3-1 \\
$v = \sqrt{v_x^2 + v_y^2}$


\vspace{2mm}
\subsection{Section 13: Forces and Torques on Currents}
\subsubsection{Pre-lecture}

\noindent
Net force acting on current \\
$\vec{F} = \sum^{}_{i} \vec{F}_i$ \\
$\vec{F} = q*\sum^{}_{i} \vec{v}_i * \vec{B}$ \\
$\vec{F} = q(N\vec{v}_{avg}) * \vec{B}$ \\
N = number of charge carriers n*AL \\
I = nAqv$_{avg}$ \\
$\vec{F} = qnAL\vec{v}_{avg}*\vec{B}$ \\
Force on Current-Carrying Wire \\
$\vec{F} = I\vec{L}*\vec{B}$

\vspace{2mm}

\noindent
$\vec{F}_{wire} = I\vec{L}*\vec{B}$ \\
Force on a closed loop is 0 \\
$\vec{F}_{Loop} = I(0)*\vec{B} = 0$ \\
$\vec{F}_{Closed Loop} = 0$

\vspace{2mm}

\noindent
Torque stuff \\
$\vec{\tau} = \vec{r} * \vec{F}$ \\
Magnitude \\
$\tau = rF sin(\theta)$ \\
$\tau_{total} = \frac{h}{2}(2IwB)sin(\theta)$ \\
$\tau_{loop} = IwhBsin(\theta)$ \\
Generalized form \\
$\tau_{loop} = IAB sin \theta$

\vspace{2mm}

\noindent
Magnetic Dipole Moment \\
$\vec{\mu} = I\vec{A}$ \\
If there are many turns, like a coil
$\vec{\mu} = NI\vec{A}$ \\
Torque on a loop \\
$\vec{\tau} = \vec{\mu}*\vec{B}$ \\
W = $\int_{\theta_1}^{\theta_2}(-\mu B sin \theta) d\theta$ \\
$\Delta U = \int_{\theta_1}^{\theta_2} \mu B sin \theta d \theta$ \\
$U(\theta) = - \vec{\mu} * \vec{B}$


\subsubsection{Homework}

\noindent
1-1, IE with consistent answer \\
$\tau = p_B * B * sin(\theta)$ \\
$p_B = I * (\frac{\sqrt(3)}{4})d^2$ \\
W = $-p_B B (cos\theta_2^0 - cos\theta_1)$ \\
$\theta_2 = 180$ \\
$\theta_1 = 0$ \\
W = $-0.25 * \frac{\sqrt(3)}{4}(0.008)^2 * 1.3 (-1-1)$ \\
W = $-1.8*10^-3$ J


\vspace{2mm}

\noindent
2-1, IE with consistent answer \\
$\frac{2 I B d N}{g}$ \\
g = 9.82 \\
convert d from cm to m \\
answer $\approx$ 0.74786

\vspace{2mm}

\noindent
3-1 \\
$\mu = IA = IWH $ \\
$\mu_x = -\mu sin \theta = -IWH sin \theta$ \\
3-2 \\
$\mu_y = \mu cos \theta = IWH cos \theta$ \\
3-3 \\
$\vec{\tau} = \vec{\mu} x \vec{B}$ \\
$\tau_z = -\mu B sin \theta = -IWHB sin \theta = \mu_x B$ \\
3-4 \\
$F_{bc} = IHB $ \\
$\tau = IWHB sin \theta $ \\
$\tau = WF sin \theta $ \\
F = IHB \\
3-5 \\
C, throw a dart...

\vspace{2mm}

\noindent
4-1 \\
$F_{ac} = IL_{ac}B = IB \sqrt{L_{ab}^2 + L_{bc}^2}$ \\
$F_{ac,x} = -IBL_{bc}$ \\
Convert L to m \\
Convert I to A (I*10$^{-3}$) \\
4-2 \\
$F_{ac,y} = F_{ac}cos \theta = IBL_{ab}$\\
4-3 \\
$\Delta U_{12} = I L_{ab} L_{bc} B$ \\
4-4 \\
A \\
4-5 \\
Same as 4-3

\subsection{Section 14: Biot-Savart Law}
\subsubsection{Pre-lecture}
\noindent
Biot-Savart Law \\
$d\vec{B} = \frac{\mu_{0}I}{4 \pi} \frac{d\vec{s}*\hat{r}}{r^2}$ \\
B = $\frac{\mu_0 I}{2 \pi R}$ \\
$\vec{F}_1 = -\vec{F}_2$  \\
$F_2 = F_1 = \frac{\mu_0}{2 \pi d}I_1 I_2 L$ \\
$B_{center} = \frac{\mu_0 I}{2R}$ \\
$B = \frac{\mu_0 I}{2} \frac{R^2}{(R^2+z^2)^{\frac{3}{2}}}$

\subsubsection{Homework}
\noindent
Preamble \\
My work assigns $x_i$ values based on order of appearance \\
The first x,y coordinates are $x_1,y_1$ etc. \\
Convert distances to m, and when d is referred to it's $\vert x_1 \vert + x_2 $ \\
It too is in meters. \\
The constant $\mu_0 = 1.25663706 * 10^{-6}$ m kg $s^{-2} A^{-2}$ \\
I don't know where it's from, but every example I found used it. \\
1-1 \\
$B_x(0,0) = \frac{\mu I_3}{2 \pi y_3}$ \\
1-2 \\
$B_y(0,0) = \frac{-\mu}{\pi d}(I_1 + I_2)$ \\
1-3 \\
$B_y(1) = \frac{- \mu I_1}{2 \pi d}(I_3 sin(30) + I_2$ \\
DEGREES \\
1-4 \\
$F_y(1) = \frac{- \mu I_1 I_3}{2 \pi d}cos(30)$ \\
1-5 \\
$F_x(2) = \frac{\mu I_2}{2 \pi d}(I_1 - I_3 sin(30))$ \\
1-6 \\
C

\vspace{2mm}

\noindent
2-1 \\
F = $IL_xB$ \\
$F_{ad,x} = I_2 H \frac{\mu I_1}{2 \pi L}$ \\
2-2 \\
$F_{bc,x} = -I_2 H \frac{\mu I_1}{2 \pi (L+W)}$ \\
2-3 \\
$F_{net,y} = 0$ \\
2-4 \\
B \\
2-5 \\
$I_3 = 2I_1 \frac{2L+W}{L+W}$

\subsection{Section 15: Ampere's Law}
\subsubsection{Pre-lecture}
\noindent
Ampere's Law \\
$\oint \vec{B} * d \vec{l} = \mu_0 I_{enclosed}$ \\
= $\frac{\mu_0 I}{2 \pi R}(2 \pi R)$ \\
= $\mu_0 I$

\vspace{2mm}

\noindent
Magnetic Field instide of a wire\\
Ampere's Law, cylindrical case \\
B(2 $\pi$ R) = $\mu_0$ I \\
B(2 $\pi$ R) = $\mu_0$ I * $\frac{\pi r^2}{\pi a^2}$ \\
B = $\frac{\mu_0 I}{2 \pi a^2} r$ for r $<$ a \\
B = $\frac{\mu_0 I}{2 \pi r}$ for r $>$ a

\vspace{2mm}

\noindent
Infinite sheets of charge \\
y components cancel, x components add \\
$\oint \vec{B}*d\vec{l} = 2BL$ \\
$I_{enclosed} = (number of wires) * I$ = nLI \\
B = $\frac{1}{2} \mu_0 n I$

\subsubsection{Homework}
\noindent
Interactive Example \\
Assume B(r) \\
Remember that $\mu_0 = 1.25663706 * 10^{-6}$ \\
B = $\mu_0 I_{enclosed}$ \\
B = $\mu_0 (I_1 - I_2 *\frac{\pi (r^2 - b^2)}{\pi (c^2 - b^2)})$ \\
Factor $\pi$ out \\
B = $4.8x10^{-7}$ T \\
$B_x = -B sin 30 = -2.4x10^{-7} $ T

\vspace{2mm}

\noindent
Section 1 \\
Found a better definition of $\mu_0$ as $4\pi * 10^{-7}$ \\
Mind your directions, and remember the right hand rule is your friend \\
I'll include what directions I had since it's probably the same for you \\
2-1 \\
R hand rule, B is only in y direction \\
$B_y = \vert B \vert$ \\
$B = \frac{\mu_0 (I_1 + I_2)}{2 \pi d}$ \\
2-2 \\
Section R - S = 0, $\vec{B} * d \vec{\ell} = 0$ \\
Section P - R is $\frac{1}{8}$ of the loop so $\frac{1}{8} \int \vec{B} *d \vec{\ell} = \frac{1}{8} \mu_0 I$ \\
I think it's the same for all, but mine is \\
$\frac{1}{8}(4 * \pi*10^{-7}) * (I_1-I_2) = 3.30*10^{-7}$ Tm\\
2-3 \\
$-B_T (2 \pi r) = \mu_0 I_{enc} + I_1$ \\
Negative due to right hand rule \\
I don't know why the two I values get added. \\
My initial attempt didn't work without it, and I found \\
an example that used it that worked. \\
$I_{enc} = I_2 \frac{r^2 - a^2}{b^2 - a^2}$ \\
$B_T = \frac{-\mu_0 (I_1 + I_{enc})}{2 \pi r}$ \\
2-4 \\
Negative of 2-2, same end points \\
2-5 \\
C

\vspace{2mm}

\noindent
Section 2 \\
3-1 \\
Both fields are y direction only, so $B_x$ = 0 \\
3-2 \\
Convert n to wires per meter, mine went from 18 to 1800. \\
B(2L) = $\mu_0 n L I$, no x field so B becomes \\
B = $\frac{1}{2} \mu (n I_1 + n I_2) $ \\
3-3 \\
$B_r = \frac{1}{2} \mu_0 n (I_1 - I_2)$ \\
3-4 \\
$I_{enc} = n h I_1$ \\
B = $\mu_0 I_{enc}$ \\
3-5 \\
Positive answer from 3-3 \\
My 3-3 was -0.001357, so my 3-5 is 0.001357 \\
3-6 \\
B*h = (answer from 3-2)*(h from 3-4) 

\subsection{Section 16: Motional EMF}
\subsubsection{Pre-lecture}

\noindent 
Electrodynamics \\
$\vec{E}(\vec{r}) \to \vec{E}(\vec{r},t)$ \\
$\vec{B}(\vec{r}) \to \vec{B}(\vec{r},t)$ 

\vspace{2mm}

\noindent
Faraday's Law \\
$\oint \vec{E} * d\vec{l} = -\frac{d}{dt} \int \vec{B}* d\vec{A}$ \\
Maxwell's Displacement Current \\
I = $\varepsilon_0 \frac{d}{dt} \int \vec{E}*d\vec{A}$ \\
Ampere-Maxwell Law \\
$\oint \vec{B}*d\vec{l} = \mu_0 I + \mu_0 \varepsilon_0 \frac{d}{dt} \int \vec{E} * d\vec{A}$

\vspace{2mm}

\noindent
At Equilibrium \\
E = vB \\
Potential Difference \\
$\varepsilon$ = vBL \\
Current \\
I = $\frac{vBL}{R}$ \\
Power Dissipated \\
$P_R = \frac{(vBL)^2}{R}$ \\
Power from External Agent \\
$P_{external} = \frac{v^2 B^2 L^2}{R}$

\vspace{2mm}

\noindent
EMF \\
$\varepsilon_{loop} = vL(B_{bottom} - B_{top})$ \\
$B_{top} = \frac{\mu_0 I_0}{2 \pi (R+w)}$ \\
$B_{bottom} = \frac{\mu_0 I_0}{2 \pi R}$

\vspace{2mm}

\noindent
At Equilibrium \\
E = v B cos $\theta$ \\
Motional EMF \\
$\varepsilon = \omega A B cos(\omega t)$ \\
$\Phi = \int \vec{B} * d \vec{A}$

\subsubsection{Homework}
\noindent
1-1: Interactive example \\
At t = 0 \\
R = 2W \\
W = 3 cm = .03 m \\
L = 8 cm = .08 m \\
B = 1.6 T \\
$L_B = 15 cm = .15 m$ \\
constant v = 5 $\frac{cm}{s}$ = .05 $\frac{m}{s}$ \\
Use $\varepsilon$ = vBL, or W in this case \\
Then I = $\frac{\varepsilon}{R}$ \\
Finally, use F = ILB = IWM \\
$F(0.8) = 5.76*10^{-5}$ N 

\vspace{2mm}

\noindent
2-1 \\
$\varepsilon$ = vBL = vB($S_1$) \\
Use vb($S_1$) for this part \\
$I = \frac{V}{R} = \frac{\varepsilon}{R}$ \\
Mind current direction, I got a sign error! \\
2-2 \\
$\varepsilon$ = vBL = vB($S_2$) \\
$I = \frac{\varepsilon}{R}$ \\
Mind your direction, this one was positive on mine \\
2-3 \\
Power = F*v = $\frac{V^2}{R}$ \\
F = $\frac{v B^2 S_2^2}{R} $ \\
2-4 \\
Opposite sign answer from 2-1 \\
My 2-1 was -0.00646, so my 2-4 is 0.00646 \\
2-5 \\
A

\vspace{2mm}

\noindent
3-1 \\
B = $\frac{\mu_0 I}{2 \pi d}$ \\
$\varepsilon$ = v B W \\
Mind your directions here! \\
3-2 \\
$d_1$ at $t_1$ = d - (v*$t_1$) \\
$B(d_1) = \frac{\mu_0 I}{2 \pi d_1}$ \\
$\varepsilon$ = v B($d_1$) W \\
3-3 \\
$d_2$ = L + d \\
$B_1 = \frac{\mu_0 I}{2 \pi d}$ \\
$B_2 = \frac{\mu_0 I}{2 \pi d_2}$ \\
$\varepsilon = vW(B_1 - B_2)$ \\
I = $\frac{\varepsilon}{R}$ \\
3-4 \\
C \\
3-5 \\
$d_2 = d + W$ \\
Use the I from 3-3! \\
For those who care, the final function is \\
$I_2 = \frac{2 \pi (R I)}{v L \mu_0 (\frac{1}{d_1} - \frac{1}{d_2})}$ \\
Working backwards \\
I = $\frac{v L \mu_0}{2 \pi R} I_2 (\frac{1}{d_1} - \frac{1}{d_2})$ \\
I = $\frac{v L}{R} B_1 - B_2$ \\
$B_1 = \frac{\mu_0 I_2}{2 \pi d_1}$ \\
$B_2 = \frac{\mu_0 I_2}{2 \pi d_2}$

\subsection{Section 17: Faraday's Law}
\subsubsection{Pre-lecture}
\noindent
Faraday's Law \\
$\varepsilon_{induced} = - \frac{d \Phi_B}{dt} $

\vspace{2mm}

\noindent
Motional EMF \\
$\vert \varepsilon \vert = \frac{d \Phi_B}{dt}$ \\
$\vert \varepsilon \vert = vl(B_{bottom} - B_{top})$ \\
$\vert \varepsilon \vert = \frac{\Delta \Phi_B}{\Delta t}$ \\
$\vert \varepsilon \vert = \omega B A cos(\omega t)$

\vspace{2mm}

\noindent
$\Phi_B = B A sin(\omega t) $ \\
$\varepsilon = -\omega N B A cos(\omega t) $


\subsubsection{Homework}

\noindent
Section 1 \\
1-1 \\
$\Phi = \frac{\mu_0 I_1}{2 \pi} * ln(\frac{L + d}{d})$ \\
1-2 \\
$\varepsilon_1 = \frac{\Phi}{t_1}$ \\
1-3 \\
No change, it's constant \\
Answer = 0 \\
1-4 \\
A, clockwise \\
1-5 \\
$\varepsilon_4 = \frac{\Phi}{\Delta t}$ \\
$\Delta t = t_4 - t_3$

\vspace{2mm}

\noindent
Section 2 \\
2-1 \\
$\omega = \frac{2 \pi}{t}$ \\
2-2 \\
$I_{max} = \frac{\omega B A}{R}$ \\
$A = .5 * b * h$ \\
2-3 \\
$\theta = \omega t$ \\
Convert to degrees! \\
$\vert \Phi \vert = \vert B * A cos \theta \vert$ \\
2-4 \\
$I = \frac{\omega B A}{R} sin(\theta)$ \\
2-5 \\
B, $\phi_0 = 0$ and $I_0 = I_{max}$ \\
2-6 \\
C, $\phi_{max}$ remains the same and $I_{max}$ doubles


\section{AC Circuits}

\subsection{Section 18: Induction and RL Circuits}
\subsubsection{Pre-Lecture}
\noindent
Self-Inductance \\
$L \equiv \frac{\Phi_B}{I}$ \\
SI Unit \\
H (Henry) = $\frac{T-m^2}{A}$ \\
Inductor Voltage \\
$\varepsilon = -L \frac{dI}{dt}$ 

\vspace{2mm}

\noindent
The Solenoid \\
Magnetic field of a Solenoid \\
$B = \mu_0 n I$ \\
Magnetic flux of a Solenoid \\
$\Phi_B = \mu_0 n^2 z \pi r^2 I$\\
Self-Inductance of a Solenoid \\
$L = \mu_0 n^2 z \pi r^2 $ \\
Where n is the wire density, z is the length, r is the radius and I is the current.

\vspace{2mm}

\noindent
RL Circuit \\
With battery \\
Closing Switch, at t = 0 \\
$V_R(0) = 0$ \\
$V_L(0) = V_b$ \\
$I_1(0) = 0 $ \\
Switch is closed for a long time, t $\to \infty$ \\
$V_R (\infty) = V_b $ \\
$V_L (\infty) = 0 $ \\
$I_1 (\infty) = \frac{V_b}{R} $
Without battery \\
Opening switch, t = 0 \\
$V_R(0) = V_b $ \\
$V_L(0) = -2V_b$ \\
$I(0) = \frac{V_b}{R}$ \\
Switch is closed, t $\to \infty$ \\
$V_R (\infty) = 0$ \\
$V_L (\infty) = 0$ \\
$I(\infty) = 0 $

\vspace{2mm}

\noindent
RL Circuit: Quantitve Description \\
Kirchoff's Voltage Rule for RL circuits \\
$L \frac{dI_1}{dt} + I_1 R - V_b = 0$ \\
redefine $\tau = \frac{L}{R}$ \\
Where L is the length of the inductor and R is the resistance of the resistor \\
$I_1(t) = \frac{V_b}{R}(1-e^{\frac{-R_t}{L}})$ \\
Yes that's t not $\tau$ \\
$V_L = V_b e^{\frac{-R_t}{L}} $

\vspace{2mm}

\noindent
Energy in an inductor \\
Inductor energy \\
$U_L = \frac{1}{2} L I^2 $ \\
Capacitor energy \\
$U_C = \frac{1}{2} \frac{Q^2}{C}$
Magnetic Energy Density \\
$u_B = \frac{B^2}{2 \mu_0}$

\subsubsection{Homework}
\noindent 
Interactive Example \\
$4.88*10^-6$

\vspace{2mm}

noindent
1-1 \\
$I = \frac{V}{\sum R}$ \\
1-2 \\
$I = \frac{V}{R_1+R_4}$ \\
1-3 \\
$ V_L = V_{R23} = I * (R_2 + R_3)$\\
1-4 \\
Yours might be different, but try $\frac{V}{R_2+R_4}$, if nothing else try .139A since it might be the same \\
1-5 \\
Same as 1-1 

\vspace{2mm}

\noindent
2-1 \\
$I_1 = \frac{V}{R_1 + R_3 + R_4}$ \\
2-2 \\
Mind your rounding here! \\
$R_{23} = (\frac{1}{R_2} + \frac{1}{R_3})^{-1}$ \\
$R_{1234} = R_1 + R_{23} + R_4$ \\
$I = \frac{V}{R_{1234}}$ \\
2-3 \\
$V_2 = V_{23} = I R_{23}$ \\
$I_2 = \frac{V_2}{R_2}$ \\
2-4 \\
Mind your directions here! Mine was a negative. \\
Also convert your L from mH to H, ex 369 to .369 \\
R = $R_2 + R_3$ \\
$I(t) = I_2 e ^{\frac{-t_{open}R}{L}}$ \\
2-5 \\
Use $I_1$ from 2-1! \\
$V_L = V_{R3} = I_1 R_3$ \\
2-6 \\
$V_L = V_R = I_2 (R_2 + R_3)$


\subsection{Section 19: LC and RLC Circuits}
\subsubsection{Pre-Lecture}
\noindent
LC Circuits: A qualitative description \\
Oscillator made by a capacitor and inductor \\
LC Circuits: A quantitive Description \\
Q(t) = $Q_{max} cos(\omega t + \phi)$ \\
Defined by inital conditions \\
$Q_{max} = CV$ \\
$\phi$ also determined, explanation given doesn't fit all 

\vspace{2mm}

\noindent 
LC Circuits: Part 3 \\
$\omega = \frac{1}{\sqrt{LC}}$ \\
LC Circuits and Energy \\
I = $-\omega Q_{max} sin(\omega t + \phi)$ \\
RLC Circuits \\
Rate of Energy Loss \\
P = $I^2 R$ \\
$\beta = \frac{R}{2L} $ \\
$\beta$ can also equal $\omega_0$ \\
$\omega'^2 = \omega_0^2 - \beta^2$ \\
$Q(t) = Ae^{-\beta t} cos(\omega' t + \phi) $

\subsubsection{Homework}
\noindent
IE 1 \\
UNITS UNITS UNITS MY DUDE!!! \\
Convert uF to F (C)\\
Convert mH to H (L)\\
Convert mA to A (I)\\
$Q_{max} = \sqrt{2 C L I^2}$ \\
If you come down with Whogivesashit, answer is $1.54 * 10^{-5}$

\vspace{2mm}

\noindent
IE 2 \\
U = $\frac{1}{2}LI^2_L + \frac{1}{2}CV^2_C$ \\
U = $\frac{1}{2}((.008)(.0293)^2+(250*10^{-6})(6.44^2))$ \\
U = 0.005188


\vspace{2mm}

\noindent
Mind your rounding and units here! \\
Convert $\mu F$ to F \\
Convert mA to A \\
Convert mH to H \\
Convert mS to S \\
1-1 \\
C = $(\frac{1}{C_1} + \frac{1}{C_2})^{-1}$ \\
$\omega_0 = \frac{1}{\sqrt{LC}}$ \\
1-2 \\
Mind your rounding on this one, also whitespace \\
$Q_{max} = \frac{I_L}{\omega_0}$ \\
1-3 \\
$V_{bc} = -L \omega_0 I_L sin(\omega t)$ \\
1-4 \\
$Q_{max} = \frac{I_L}{\omega_0}$ \\
1-5 \\
D, $Q_1 =0 and V_L = 0$

\vspace{2mm}

\noindent
Convert MH to H \\
Convert $\mu$F to F \\
2-1 \\
$I = \frac{V}{R_1}$ \\
$U_1 = \frac{1}{2} L_1 I^2$ \\
2-2 \\
L = $L_1 + L_2$ \\
$\omega_0 = \frac{1}{\sqrt{L C}}$ \\
2-3 \\
$Q_{max} = \frac{I}{\omega_0}$ \\
2-4 \\
$-Q_{max} \sin(\omega t)$ \\
2-5 \\
Convert answer to mS from S\\
$t_2 = \frac{pi}{2 \omega_0}$ \\
2-6 \\
$U_{total} = \frac{1}{2} L I^2$

\subsection{Section 20: AC Circuits}
\subsubsection{Pre-Lecture}
\noindent
Induced Voltage \\
$\varepsilon(t) = \varepsilon_m sin(\omega t)$ \\
I = $I_m sin (\omega t - \phi)$ \\
KVR = $V_c - \varepsilon$ = 0 \\
$V_R = I_R R = \varepsilon_m sin (\omega t) $ \\
$I_R = \frac{\varepsilon_m}{R} sin(\omega t)$ \\
Question 1, The voltage is positive and decreasing \\
$\frac{Q}{C} = \varepsilon_m sin(\omega t) $ \\
Q = $C \varepsilon_m sin (\omega t)$ \\
$I_C = \frac{dQ}{dt} = \omega C \varepsilon_m cos (\omega t)$ \\
Reactance of the Capacitor \\
$X_C \equiv \frac{1}{\omega C}$ \\
$I_L = -\frac{\varepsilon_m}{X_L} cos (\omega t)$ \\
Reactance of the inductor \\
$X_L \equiv \omega L$ \
tan $\phi = \frac{X_L - X_c}{R}$ \\
$I_m = \frac{\varepsilon_m}{\sqrt{R^2 + (X_L - X_c)^2}}$ 

\subsubsection{Lecture notes}
\noindent
$\omega = \frac{2 \pi}{T}$ \\
Amplitude = $ \frac{V_{max}}{R}$ \\
$\varepsilon(t) = V_{max} sin(\omega t)$  \\
$V_{max} = I_{max} * \frac{1}{\omega C} $ \\
$V_C = I_C X_C$ \\
$X_C = \frac{1}{\omega C} $
Inductor \\
KVR: $V_L - \varepsilon = 0$ \\
$\int L = \int \varepsilon(t) = \int V_{max} sin(\omega t) dt$ \\
$V_L = I_L X_L = I_L \omega L$ \\
Voltage lead \\
L, $V_L$ leads I \\
Voltage Lags \\
C, $V_C$ lags I \\
$V_L + V_R = \varepsilon$ \\
$V_L = I \omega L + IR = V_{max} sin(\omega t)$ \\
Currents $I_L$ and $I_R$ are the same

\subsubsection{Homework}

\noindent
IE1 \\
$\phi = \tan^{-1} (\frac{X_L - X_C }{\frac{\varepsilon^2}{I^2} - (X_L -X_C)})$ \\
$\phi = 23.30575$

\vspace{2mm}

\noindent
Convert C from $\mu$F to F \\
Convert L from mH to H \\
1-1 \\
Z = $\sqrt{R^2 + (X_L - X_C)^2}$ \\
Z = $\sqrt{R^2 + (\omega L - \frac{1}{\omega C})^2}$ \\
1-2 \\
$I_{max} = \frac{\varepsilon_m}{Z}$ \\
1-3 \\
$\phi = \tan^{-1}(\frac{\omega L - \frac{1}{\omega C}}{R})$ \\
MIND YOUR SIGNS!!! \\
1-4 \\
Convert $\omega$ to degrees per second \\
$\frac{90^o + \phi}{\omega} = t$ \\
1-5 \\
B \\
1-6 \\
$V_{C,max} = \frac{I_{max}}{\omega C}$ \\
$V_C(t) = V_{C,max} \cos(\phi)$

\vspace{2mm}

\noindent
2-1 \\
In these examples, use $\phi$ instead of the $\varphi$ \\
Convert L from mH to H \\
Convert C from $\mu$F to F \\
$t = \frac{\phi}{\omega}$ \\
2-2 \\
Z = $\frac{R}{\vert \cos{\phi} \vert}$ \\
2-3 \\
L = $\frac{R \tan(\phi) + \frac{1}{\omega C}}{\omega}$ \\
Convert resulting value to mH \\
2-4 \\
$I_{L,max} = \frac{\varepsilon_m}{Z}$ \\
$V_{L,max} = I_{L,max}(\omega L) = \frac{\varepsilon_m \omega L}{Z}$ \\
2-5 \\
$V_{L,max} = V_{L,max} cos(\phi)$ \\
2-6 \\
A

\subsection{Section 21: AC Circuits: Resonance and Power}

\subsubsection{Pre-Lecture}
\noindent
Reactance \\
Inductor:$X_L = \omega L$ \\
Capacitor: $X_C = \frac{1}{\omega C}$ \\
Maximum Voltages \\
Inductor: $V_L = I_m X_L$ \\
Capacitor: $V_C = I_m X_C$ \\
Resistor: $V_R = I_m R$ \\
Impedance \\
Z = $\sqrt{R^2 + (X_L - X_C)^2}$ \\
Phase \\
tan $\phi = \frac{X_L - X_C}{R}$ \\
Maximum Current \\
$I_m = \frac{\varepsilon_m}{Z}$ \\
$I_m = \frac{\varepsilon_m}{\sqrt{R^2 + (\omega L - \frac{1}{\omega C})^2}}$\\
$\omega_0^2 = \frac{1}{L C}$
Question 1 \\
C, it will decrease 

\vspace{2mm}

\noindent
$I_m = \frac{\varepsilon_m}{R} \cos(\phi)$ \\
$\cos(\phi) = \frac{R}{\sqrt{R^2 + (X_L - X_C)^2}}$ \\
x $\equiv \frac{\omega}{\omega_0}$ \\
$Q^2 \equiv \frac{L}{R^2 C}$ \\
$I_m = \frac{\varepsilon_m}{R} \frac{1}{\sqrt{1 + Q^2 \frac{(x^2 - 1)^2}{x^2}}} $ \\
btw fuck these symbols... \\
$\langle P_{Generator} \rangle = \langle P_{Resistor} \rangle = \frac{1}{2} I_m \varepsilon_m \cos \phi$ \\
Average Power per cycle \\
$\langle P_{Generator} \rangle = \varepsilon_{rms} I_{rms} \cos \phi = \frac{\varepsilon_{rms}^2}{R} \cos^2 \phi $ \\
$\langle P_{Generator} \rangle = \frac{\varepsilon_{rms^2}}{R} \frac{x^2}{x^2 + Q^2(x^2 - 1)^2}$ \\
$\varepsilon_{rms} = \frac{\varepsilon_m}{\sqrt{2}}$ \\
$I_{rms} = \frac{I_m}{\sqrt{2}}$ \\
Quality Factor \\
Q $\equiv 2\pi [\,\frac{U_{max}}{\Delta U} ]\,_{cycle}$ \\
$Q^2 = \frac{L}{R^2 C}$ \\
$V_{L_{max}} \vert_{\omega = \omega_0} = V_{C_{max}} \vert_{\omega = \omega_0} = Q \varepsilon_m$ \\
Question 2 \\
A, 0 Volts

\vspace{2mm}

\noindent
Ideal Transformer \\
$V_S = \frac{N_S}{N_P} V_P$ \\
$I_P = \frac{N_S}{N_P}I_S$

\subsubsection{Lecture notes}
\noindent
Cannot operate at different frequencies at once \\
90 degrees for all lines is always the same and do not change  relative \\
$I(t) = I_m \sin(\omega t - \phi)$ \\
Z = $\frac{\varepsilon_{max}}{I_{max}}$ \\
Always out of phase by 90 degrees \\
Current remainds constant \\
In phase with voltage drop of the resistor \\
$I_{\varepsilon}(t) = I_R(t) = I_L(t) = I_C(t)$ \\
All out of phase with the current 

\vspace{2mm}

\noindent
Slides, "Ohms" Law for each element \\
ONLY for maximum values \\
RMS = Root mean square \\
$I_{peak} = I_{max}\sqrt{2} $ \\
$\langle I^2 R \rangle = I_{max}^2 R$ \\
There will be a Phasor question in the exam \\
Redefinition of $\varepsilon$ = $I_m Z$
Wanna deliver 1500 W, power. 100 Volts over transmission lines, resistance R = 5 ohms. \\
How much power is lost in the lines? \\
Power = 1500 W \\
Calculating power lost in the lines \\
Power dissapated = $P_R = IV = 20 * 100 V = 2000W$, first attempt not right \\
I = $\frac{V}{R} = \frac{100V}{5 \Omega} = 20 A$ \\
$P_R = IV = I(IR) = I^2R = 15^2A 5 \Omega = 1125W$ \\
$I_{10kv} = \frac{1500}{10000} = 0.15A$ \\
$(.15A)^2 * (5 \Omega) = 0$ \\
High voltage low current prevents loss 

\subsubsection{Homework}
\noindent
IE \\
I'm not going to show the functions for this, since the functions aren't used later \\
If you want the functions, let me know and I'll add them from now on! \\
0.000839 

\vspace{2mm}

\noindent
I use $\phi$ instead of $\varphi$ for consistency in notes. \\
1-1 \\
$I_m = \frac{2 P_{avg}}{\varepsilon_m cos \phi}$ \\
1-2 \\
R = $\frac{2 P_{avg}}{I_m^2}$ \\
1-3 \\
C = $(\omega^2 L - \omega R \tan(-\phi))^{-1}$ \\
1-4 \\
B \\
1-5 \\
$P_{avg} = \frac{1}{2} \frac{\varepsilon_m^2}{R}$

\vspace{2mm}

\noindent
2-1 \\
L = $\frac{1}{\omega^2 C}$ \\
Answer in H, convert H $\to$ mH!!! \\
2-2 \\
$U_{max} = \frac{1}{2} L I_m^2$ \\
Multiply answer by $10^{-3}$ to convert to Joules! \\
My answer went from 49.46 $\to$ 0.0495 \\
2-3 \\
$\Delta U = \frac{\pi \varepsilon_m I_m}{\omega}$ \\
2-4 \\
Q = $\frac{U_{max}}{\Delta U} 2 \pi$ \\
2-5 \\
R = $\frac{\varepsilon_m}{I_m}$ \\
2-6 \\
B

\section{Light and Optics}

\subsection{Section 22: Displacement Current and Electromagnetic Waves}
\subsubsection{Pre-Lecture Notes}
\noindent
Modified Ampere's Law \\
$\oint \vec{B} * d \vec{l} = \mu_0 I + \mu_0 \varepsilon_0 \frac{d \phi_E}{dt}$ \\
$Q = \varepsilon_0 \phi_E$ \\
Displacement Current \\
$I_D = \varepsilon_0\frac{d \phi_E}{dt}$ \\
h(x,t) = $A \cos(kx - \omega t)$ \\
Harmonic Plane Wave \\
Amplitude A \\
Wave number k = $\frac{2 \pi}{\lambda}$ \\
Angular Frequency $\omega \frac{2 \pi}{T}$ \\
Period T \\
Frequency f = $\frac{1}{T}$ \\
Velocity v = $\lambda f = \frac{\omega}{k}$

\vspace{2mm}

\noindent
Wave Equation \\
$\frac{\partial^2 E_x}{\partial z^2} = \mu_0 \varepsilon_0 \frac{\partial^2 E_x}{\partial t^2}$ \\
Speed of Electromagnetic Wave \\
v = $\frac{1}{\sqrt{\mu_0 \varepsilon_0}}$
A harmonic solution \\
$E_x = E_0 cos(k z - \omega t)$ \\
$B_y = \frac{k}{\omega} E_0 cos (kz - \omega t)$ \\
$B_y$ is in phase with $E_x$ \\
$B_0 = \frac{E_0}{c}$

\subsubsection{Homework}
1-1 \\
Convert $\lambda$ from nm $\to$ m \\
k = $\frac{2 \pi}{\lambda}$ \\
1-2 \\
$Z_{max} = \frac{\pi}{2k}$ \\
Convert answer from m $\to$ nm \\
1-3 \\
c is the speed of light \\
c = $3*10^8 \frac{m}{s}$ \\
$E_{max} = \sqrt{2} c B1$ \\
1-4 \\
Mind your directions here! Mine was negative! \\
$E_y = c B_1$ \\
1-5 \\
B \\
1-6 \\
$t_{max} = \frac{\lambda}{4c}$ \\
Make sure $\lambda$ is in m \\
1-7 \\
B \\
END TEST 2!

\subsection{Section 23: Properties of Electromagnetic Waves}
\noindent
BEGIN FINAL!!!
\subsubsection{Pre-Lecture Notes}
\noindent
$E_x = E_0 \sin(kz-\omega t)$ \\
$B_y = B_0 \sin(kz - \omega t)$ \\
Velocity c = $\frac{\omega}{k} = \frac{1}{\sqrt{\mu_0 \varepsilon_0}} = \frac{E_0}{B_0}$ \\
E-M Wave Speed c = $f \lambda$ \\
Doppler Shift f' = f $\frac{1 \pm \beta}{1 \mp \beta}$ \\
Energy Densities \\
Electric Fields $u_E = \frac{1}{2} \varepsilon_0 E^2$ \\
Magnetic Fields $u_B = \frac{1}{2} \frac{B^2}{\mu_0}$ \\ 
E = cB \\
u = $\varepsilon_0 E^2$ \\
Average Energy Density \\
$\langle u \rangle = \frac{1}{2}\varepsilon_0 E^2_0$
Intensity \\
$I \equiv \frac{\langle Power \rangle}{Area}$
Poynting Vector \\
$S = c \varepsilon_0 E^2$ \\
Sunlight Electric Field Strength \\
$E_{rms}^2 = \mu_0 c I$ \\
I = 100 $\frac{mW}{cm^2}$ \\
Impedance of Free Space $Z_0 \equiv \mu_0 C$ \\
Energy = E = hf \\
Momentum = p = $\frac{E}{c}$ = $\frac{h}{\lambda}$ 

\subsubsection{Homework}
\noindent
IE \\
$B_Z = -61.36*10^{-8}$

\vspace{2mm}

\noindent 
c is the speed of light, $3*10^8$ \\
1-1 \\
f = $\frac{c}{\lambda}$ \\
1-2 \\
$\varepsilon_0 = 8.854E-12$ \\
$I = \frac{1}{2} \varepsilon_0 c^3 (B_x^2+X_y^2)$ \\
1-3 \\
$\mu_0 = 4 \pi * 10^{-7}$
$S_Z = -\frac{c}{\mu_0}(B_x^2 + B_y^2)$ \\
1-4 \\
$E_x = -cB_y$
1-5 \\
D

\subsection{Section 24: Polarization}
\subsubsection{Pre-Lecture Notes}
$E_x = E_0 \cos \theta sin(kz - \omega t + \phi)$ \\
$E_y = E_0 \sin \theta sin(kz - \omega t + \phi)$ \\
Incident Light \\
Unpolarized:  $I_{final} = \frac{1}{2}I_0 $ \\
Law of Malus \\
Polarized: $I_{final} = I_0 \cos^2 \theta$ \\
$I_1 = \frac{1}{2}I_0$ \\
$I_2 = I_1 \cos^2 30$ \\
$I_3 = I_2 cos^2 60$ \\
Linear Polarization \\
Relative Phase $\phi \equiv \phi_x - \phi_y = 0$ \\
Circular Polarization \\
Realative Phase $\phi \equiv \phi_x - \phi_y = \pm \frac{\pi}{2}$ \\
+ is right rotation \\
- is left rotation \\
$\Delta \phi = \phi_y - \phi_x = \omega d (\frac{1}{v_{fast}} - \frac{1}{v_{slow}})$\\
$\Delta \phi = \frac{\pi}{2}$\\
Remember Snell's Law and the Law of Reflection!!! 

\subsubsection{Homework}
\noindent
IE \\
$75.78^o$

\vspace{2mm}

\noindent
1-1 \\
$I_1 = \frac{I_0}{2}$ \\
$I_2 = I_1 \cos^2(\theta_1 - \theta_2)$ \\
1-2 \\
$I_{final} = I_2 \cos^2(\theta_3 - \theta_2)$ \\
1-3 \\
$I_{final,new} = \frac{1}{2} I_0 (\cos(\theta_1 - \theta_3) \cos(\theta_2 - \theta_3))^2$ \\
1-4 \\
B \\
1-5 \\
$I'_{final} = \frac{1}{2} I_0 (\cos(\theta_2 - \theta_3) \cos \theta 3)^2$

\vspace{2mm}

\noindent
2-1 \\
$I_{mid} = \frac{I_0}{2}$ \\
2-2 \\
$I_{final} = I_{mid} \cos^2 \theta_1$ \\
2-3 \\
$\frac{E_{y,final}}{E_0} = \sqrt{\frac{I_{final}}{I_0}} \sin \theta_1$ \\
2-4 \\
$I_{final,new} = I_0 \cos^2(90^o - \theta_1) \cos^2 \theta_1$ \\
2-5 \\
B \\
2-6 \\
C

\subsection{Section 25: Reflection and Refraction}
\subsubsection{Pre-Lecture}
\noindent
Law of Reflection \\
$\theta_i = \theta_r$ \\
$ct_{ab}=ct_{dc}$ \\
$v = \frac{1}{\sqrt{\mu \varepsilon}}$ \\
Index of refraction \\
$n \equiv \frac{c}{v} = \frac{\sqrt{\mu \varepsilon}}{\sqrt{\mu_0 \varepsilon_0}}$ \\
n $\approx \sqrt{\frac{\varepsilon}{\varepsilon_0}}$ \\
Snell's Law \\
$n_2 sin \theta_2 = n_1 sin \theta_1$ \\
Glancing incidence \\
$\theta$ ~ $90^o$ \\
Normal incidence \\
$\theta = 0^o$ \\
R = $(\frac{n_2 - n_1}{n_2 + n-1})^2$ \\
Critical Angle \\
$\theta_c = \sin^{-1}(\frac{n_2}{n_1})$ \\
Brewster's Angle \\
$\tan \theta_1 = \frac{n_2}{n_1}$
\subsubsection{Homework}
\subsection{Section 26: Lenses}
\subsubsection{Pre-Lecture}
\noindent
Converging lenses are convex \\
Diverging lenses are concave \\
Lense Equation \\
$\frac{1}{s} + \frac{1}{s'} = \frac{1}{f}$ \\
s' = $\frac{f s}{s-f}$ \\
Magnification \\
M $\equiv \frac{h'}{h} = -\frac{s'}{s}$\\
M = $\frac{-f}{s-f}$ \\
Lensmaker's Formula \\
$\frac{1}{f} = (n-1)\frac{1}{R}$

\subsubsection{Homework}
\subsection{Section 27: Mirrors}
\subsubsection{Pre-Lecture}
\noindent
The mirror equation \\
$\frac{1}{s} = \frac{q}{s'} = \frac{1}{f}$\\
Magnification \\
M $\equiv \frac{h'}{h} = -\frac{s'}{s}$\\
s' = $\frac{fs}{s-f} \lt 0$\\
M = $-\frac{f}{s-f} \gt 0$ \\
Plane Mirrors \\
s' = -s \\
h' = h \\
M = 1 \\
Small angle approximation \\
f = $\frac{R}{2}$
\subsubsection{Homework}
\subsection{Section 28: Optical Instruments}
\subsubsection{Pre-Lecture}
\subsubsection{Homework}

\section{The final review lecture}
\noindent
Extra credit for recitation tomorrow \\
Sorry no pictures... \\
What is he direction of the current induced around the wire loop at time t =5 sec? \\
clockwise or counterclockwise? Clockwise \\
Induced current must produce positive B field

\vspace{2mm}

\noindent
Calculation problem \\
Magnitude of the current induced around the wire loop at time t = 5 sec: \\
1.3 mA, 2.0 mA, 9.0 mA, 12 mA, 18 mA \\
$I_{ind}R = emf_{ind}$ \\
$I_{ind} = \frac{emd_{inf}}{R} = \frac{1}{R} \frac{d\Phi_B}{dt} = \frac{1}{R} A \frac{dB}{dt}$ \\
$I_{ind} = \frac{wh}{R}(\frac{-4T}{2s}) = \frac{1.5m*/06m}{150 \omega} * 2 \frac{T}{s} = -1.2*10^{-2} A$ \\
Loop moving \\
$Ii_{ind} = \frac{B}{R} \frac{d}{dt} A(t) = \frac{-Bhv}{R}$ \\
A(t) = h w(t) = h ($\frac{1}{2}w - vt$) \\
$\frac{dA}{dt}$ = hv 

\vspace{2mm}

\noindent
More calculations \\
$I_{ind} = \frac{1}{R} \frac{d}{dt} \phi_B = \frac{h}{R} ((mt+x)(-v)+(w-vt)m) = \frac{h}{R} ((-vm-vm)t-nv+wm)$ \\
$\phi_B =\int \vec{B}*d\vec{A} = (mt+n)*h*(w-vt)  = h(mt+n)(w-vt)$ \\
emf = $\frac{d\phi_B}{dt}$

\vspace{2mm}
Phasors \\
$\beta + \frac{\pi}{2} = \alpha + \omega t_2 \to \alpha = \beta$

\end{document}
